
\section{Introduction}\label{sec:introduction}
In the beginning of the project, our supervisor decided that the project group was to be divided into a software group and a hardware group. Our supervisor also assigned four roles of leadership as follows: One overall project manager, one manager for the hardware group (me), a manager for the software group and one person responsible for documentation. \newline
The person assigned as software manager was also assigned to be responsible for sponsorship, which in practice came to mean that he devoted the wast majority of his time to sponsorship and consequently there was no software manager. I will come back to this later. \newline
The role of the person responsible for documentation was pretty much abolished as well and the responsibility of documentation was spread over the whole group. This did however not give any apparent negative effects so I will not go into more detail about this.

One of the first things the four "managers" did was that we interviewed each of the other members of the group to learn who they were, what they were good at and if they wanted to be put in the software or hardware group. These interviews were very good since they gave a good overview of what people there were in the group and which role should be given to whom. \emph{I strongly recommend the managers of the next year's project to do the same.}

One open question that was not decided by the supervisor was whether \emph{firmware} was to be considered to be software (and consequently the responsibility of the software group) or hardware (the responsibility of the hardware group). This question, and the question where to draw the line between firmware and software, was to give significant problems to the project. \newline
One obvious way would be to say that any programming (software or firmware) would be done by the software group. The problem with this way would have been that many parts of the robot include both hardware and software. For example, the INS controller (that handles the inertial measurement unit and fiber optic gyroscope) consists of both hardware (the sensors themselves and interface circuits) and also firmware for reading the sensors, sending CAN messages and so forth. In the light of this fact, putting the hardware development and firmware development in two different groups did not look like the best of ideas.

Firmware was decided to be under the responsibility under the hardware group. As for the division between software and firmware, the Mission control, Motion control, Simulator, the vision system was considered to be software and more or less any other form of programming was considered firmware. \newline
I did in initially support this, mainly since I foresaw that there would be many times when hardware and firmware needed to be developed together and that there would often be significant interdependence between hardware and firmware. Also, I believe that some part of my ego also wanted me to have the responsibility (and therefore reasons to take credit) for work that also contained some programming. \newline
However, as the project went on, I came to regret this and I realized that the division of work should have been done in a different way.

\newpage