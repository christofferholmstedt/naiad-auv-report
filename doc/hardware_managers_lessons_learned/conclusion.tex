\section{Conclusion}\label{sec:conclusion}
Project courses such as this project course are good since they let each student use his/her skills in a realistic context, cooperate with others and do something that has not been done before. Because of this it is important that each student has a solid base of skills in his/her area(s) before starting such a project and that cooperation within the project group works well.

As for my advice for a plan for future projects I recommend to use the two weeks for analyzing the requirements given by the customer and for State-of-the-Art research. Two weeks for defining the structure of the robot, its subsystems and the interfaces between them. Throughout this process one should keep a close contact with the customer (the person(s) or entity that orders the robot) in order to make sure that the structure decided upon is what the customer wants. \\
Once all subsystems are defined, at least one person should be assigned as responsible for each subsystem (one person can be responsible for several subsystems).

The next phase of the project would then be to define the structure of each subsystem in order to get an initial understanding for how each subsystem should be implemented. When this has been done it might very well become apparent that one or several subsystems can not be implemented in the intended way and/or that the interface of a subsystem has to be changed.\emph{ If this happens, then this would be something good.} The fault in the overall structure of the robot can then be detected at an early state before any major implementations have been done and consequently lots of time can be saved this way. \\
After this phase comes the actual implementation phase. Testing should be done continuously so that any errors or flaws are detected as early as possible. When several subsystems that are to work together have been implemented, one should try to test them together to verify that their common interface works (that they work together). As more and more subsystems are finished, more and more complex (and accurate) tests can be done.

Lastly, remember to document all work in a proper way. Your work is of little use if nobody else can understand or use the robot(s) that you have built during the project.