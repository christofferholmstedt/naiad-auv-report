% Project Management

% Do NOT change this "Section" title
% and do NOT add more "Section" level titles.
\section{Lessons learned}\label{sec:learnings}
This part will mostly be directed to the next project manager and should be considered as subjective advices rather than lessons.

\subsection{"Less is more" does not apply to budgets}
Rough estimates indicated that we only had about one fifth of the amount of money that we would need in order to finish the project -- and still meet all the requirements. Needless to say, this was something that we had to take care of. Therefore a role as a full-time sponsor seeker was created and filled. After a while the budget had more than tripled -- taking into account the cost of the components that some sponsors gave. With this in mind, there is no way that I would not recommend you to create and fill this position.
\subsection{Things never turn out the way you imagined}
Less than two weeks into the project, the original hierarchy fell ... Our software manager became the head of external relations and the duties of the software manager were split between the two of us. Even though it worked for quite some time, in the end, the software group suffered from lack of leadership. Don't let anyone have more than one role, at least not if they're important ones.
\subsection{Stranger danger}
Here I want to stress the importance of trying to get to know each other -- and to arrange events in which it can be done smoothly. If you want people to work effectively in a group then you must see to it that they utilize the benefits of being in one. For example sharing knowledge or simply giving inputs on different matters.
Once we started to eat out on a regular basis I would say that the overall productivity increased and the working environment became a much livelier one -- to the better if you ask me. The group is greater than the sum of its individuals.
\subsection{Minorities are the loudest}
You will be in charge of a lot decisions and, regardless of your own standings, a minority will not agree to the outcome of it. This will sometimes end in heated discussions or complaints and, if you're anything like me, might be taken personal. But don't. It's, most probably, not about you but about the subject itself, and by the end of the day the complaints should rest on the shoulders of the subject and not be lurking about in the corners of your mind. Reassure yourself, if needed, that the silent majority stands behind the decision -- even though they don't sound as much.
\subsection{Learning experience versus deadlines}
There's a constant battle within myself when it comes to this subject. On the one hand this project is a course in which we are supposed to learn new things, on the other hand, there's limited time. Twenty weeks might sound like a long time, however, as the number of weeks left becomes fewer and fewer and the amount of work seems to go in the other direction, you realize it's not that long. So to summarize it, don't let the upcoming deadlines refrain you from trying out new, and possibly more suitable, solutions. But on the other hand, don't let the search for better or more suitable solutions take up too much time -- there has to be a balance between the two. I guess that is easier said than done as most of us probably want to improve our work and perhaps finds joy in optimizing, simply for the sake of optimizing.
\subsection{A saddle is not a bike -- no matter how comfortable it might be}
Everything that is created must be compatible with the rest of the project, and that is apparently easy to forget. This might occur when someone has been working on a single part for too long -- it becomes all about that part and not the big picture. If it can't be integrated with the rest of the project then it becomes useless -- at least for the project. It might sound harsh but that is my take on the subject.