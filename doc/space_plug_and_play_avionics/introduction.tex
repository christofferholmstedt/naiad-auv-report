% Do NOT change this "Section" title
% and do NOT add more "Section" level titles.
\section{Introduction}\label{sec:introduction}
As a requirement for the project, Naiad was to use
Space plug-and-play Avionics technology to make it modular in the sense that
it should be easy to add and remove parts from it for different missions. In
this report lessons learned from this work will be presented for future
interest in this area.

\subsection{Space plug-and-play Avionics}
\subsubsection{Background}
Space plug-and-play Avionics (SPA) technology, has been developed by NASA and partners to improve the time it takes to assemble a new satellite for specific missions. In other words, the goal was to decrease the development time by designing a plug-and-play system of interchangeable components in such a way that no matter the mission, a customized satellite can be put together within hours or days, and meet the specific requirements.

\subsubsection{The Standard}
SPA is specified in a set of standards documents, each standard specifying
functionality in a specific area ranging from "SPA Logical Interface Standard"
to "28V Power Service". The standard documents defines everything from power
supply levels to how components should communicate with each other.
Focus during the Naiad project have
been a subset of all the standards more specifically the following ones:

\begin{itemize}
  \item Local Subnet Adaptation Draft
  \item Logical Interface Standard \cite{spa:logical_interface}
  \item Networking Standard \cite{spa:networking}
  \item SpaceWire Subnet Adaptation Standard \cite{spa:spacewire}
  \item Mini-PnP / SPA-1 Protocol Draft
\end{itemize}

The Local Subnet Adaptation Draft defines how inter-processing communication
should be done. It defines the use of UDP/IP as the main protocols to use. The
Logical Interface Standard defines how hardware and software components should
communicate on an application level. The Networking Standard, SpaceWire Subnet
Adaptation Standard and the Mini-PnP / SPA-1 Protocol Draft all defines how the
respective subnets should work and be connected with each other.

The network
standard defines how enumeration should be done for all components in
the network while the hardware specific adaptations defines how address
resolution should be done in respective hardware specific network.

\subsection{SPA over CAN Bus}
Another requirement from the customer was the use of CAN Bus
\cite{standard:can_bus} for communication between sensors and actuators.
Therefore, a major part of the implementation was to design and develop SPA functionality over the CAN
Bus. No previous work has been found in this field.

\subsection{SPA with Ada}
A third requirement from the customer was the use of Ada as programming
language. Previous implementation of the SPA standard exists in C/C++ but not
in Ada.

\subsection{Virtual Plug-and-Play Network}
Virtual Network Protocol (VNP) \cite{web:vnp} or Virtual Plug-and-Play
Network, VPPN (working titles) is an initiative to bring SPA technology to
the consumer and business market outside of the space industry. VPPN put focus
on the software part of the SPA standard and have therefore removed all parts
related to hardware specifics such as which hardware connectors to use. To
clarify this, SPA and VPPN are (as of writing) different names for the same technology but for different business targets. Throughout this report VPPN will be used
except when referring to specific reports.

The following parts of the report is structured as follows. Chapter \ref{sec:implementation}
goes through what has been done with VPPN in the Naiad project and chapter
\ref{sec:conclusion} focuses primarily on future work.
