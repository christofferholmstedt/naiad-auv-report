% Do NOT change this "Section" title
% and do NOT add more "Section" level titles.
\section{Introduction}\label{sec:introduction}
As a requirement for the project the autonomous underwater vehicle were to use
Space plug-and-play Avionics technology to make it modular in the sense that
it should be easy to add and remove parts from it for different missions. In
this report lessons learned from this work will be presented for future
interest in this area.

\subsection{Space plug-and-play Avionics}
\subsubsection{Background}
Space plug-and-play Avionics (SPA) has been developed by NASA and partners to
improve the time it takes to assemble a new satellite for specific missions.
In other words the goal is not to decrease the development time of specific
components but instead design a plug-and-play system such that no matter what
the mission requirements are, NASA can put together a sattelite within hours or
days that meet a new mission's requirements.

\subsubsection{The Standard}
SPA is specified in a set of standards documents, each standard specifying
functionality in a specific area ranging from "SPA Logical Interface Standard"
to "28V Power Service". The standards document together defines everything from power
supply levels and which hardware connectors to use to how software components should
communicate with other software components. Focus during the Naiad project have
been a subset of all the standards more specifically the following ones:

\begin{itemize}
  \item Local Subnet Adaptation Standard
  \item Logical Interface Standard
  \item Networking Standard
  \item SpaceWire Subnet Adaptation Standard
  \item ... SPA1 / MiniPlug?
\end{itemize}

\subsection{SPA over CAN Bus}

\subsection{SPA with Ada}
