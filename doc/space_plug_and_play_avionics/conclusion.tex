% Do NOT change this "Section" title
% and do NOT add more "Section" level titles.
\section{Conclusion}\label{sec:conclusion}
1) Need Central Addressing Service.
2) Need Lookup Service.
3) Need Data Manager.
4) Need simple test service/data provider.
5) Need simple test consumer.

None of the listed parts can be created and tested without the others, they are
all needed to set up basic functionality in a VPPN network. Therefore future
work should focus on implementing this in Ada within one processing unit /
local subnet. When that is done work can continue with gateways to other
networks.



% You can use how many "subsections" and "subsubsections" you like.
\subsection{VPPN and CAN}
In section \ref{subsubsec:vppn_can_bus} the obvious problem with too few bits
available to work with was presented. Suggested work for the future is to
tackle the problem from IETFs suggestion \cite{web:draft-ip_over_can} with a much smaller prioritisation
band

What about sending a CAN Message with the a specific CAN message header and
empty payload. The CAN endpoints that transmit would not detect a
collision...or would the CRC be different? Maybe you can solve the ARP this
way.

Text
\subsubsection{Subsubsection1}
Text
