% Do NOT change this "Section" title
% and do NOT add more "Section" level titles.
\section{Conclusion}\label{sec:conclusion}
Due to limitations presented in this report Space plug-and-play Avionics (SPA) have
not been used in the final AUV. The following two sections show how to continue
with the combination of SPA/VPPN, CAN Bus and Ada.

\subsection{Virtual Plug-and-Play Network}
\subsubsection{VPPN and CAN}
In section \ref{subsubsec:vppn_can_bus} the problem with too few bits
available to work with when it comes to address resolution was presented.
Alternative solutions not explored could be to store a part of the unique
addresses in the CAN Bus Message ID and the rest in the payload. A second
alternative would be to use a randomly generated identifier and with math
prove the unlikelihood of a CAN Bus collission.

Suggested work for the future is to tackle the problem starting from the now
expired IETF draft presented by Cach and Fiedler \cite{web:draft-ip_over_can}.
The benefit of this solution is a
smaller prioritisation field which make it possible to use more bits for
addressing during the address resolution phase. Though worth remembering is
that the solution presented by Cach and Fiedler is not designed for systems
with hard real-time requirements.

\subsubsection{VPPN and Ada}
The main lesson learned during development of the minimal implementation
presented in this report was the lack of VPPN infrastructure
components to test with. Most notable the Central Addressing Service (CAS) and the
Lookup Service (LS). If UDP/IP were used previous implementation in C/C++ could be
used as drivers/stubs while developing the needed infrastructure in Ada. As
this is not the case most reasonable approach to implement VPPN in Ada would
be to start with one local subnet within one single processing unit.

First step would be to implement the CAS, the LS and the Local Subnet Manager (SM-L).
After that a test service provider and test application could be implemented
within the same subnet before moving on with VPPN Gateways to other subnets.

The reason for implementing the CAS, LS and SM-L in one go is
that without CAS the SM-L can't give out new addresses on the local subnet and
without the SM-L the CAS can't be connected to the local subnet (without
hardcoding address information). The LS is then needed before adding
applications and service providers so the VPPN network can handle actual data.
