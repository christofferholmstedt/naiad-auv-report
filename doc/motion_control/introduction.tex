% Do NOT change this "Section" title
% and do NOT add more "Section" level titles.
\section{Introduction}\label{sec:introduction}

This report will cover the motion control system in the Naiad AUV. Then Naiad UAV is a robotics project at Mälardalens University autumn of 2013. The goal of the project is to build a autonomous underwater vehicle for salvaging of industrial waste in BOTTENVIKEN. 

In this report we will go over the reasoning and the implementation of the motion control system for the Naiad AUV. 

The motion control system for the Naiad AUV is a layerd software system which has its basis in PID control. Each component of the current location(orientation and position) of the craft is controlled by a PID controller. 

All PID based calculations are based in a optimal thruster configuration where each thruster only affects one component.

A conversion from the optimal to the real system is done through Gaussian elimination on a matrix that represents each thrusters effect on each component in the actual craft.

With this approach the flow of data in the system is easy to follow and it provide the Naiad AUV with a reliable motion control system that executes its task flawlessly with proper configuration.

The report is organized as follows, in section 2 the methodology used is explained. In section 3 the implementation of the system in question is explained. In section 4 the results are shown and in section 5 the authors conclusions are given. 
