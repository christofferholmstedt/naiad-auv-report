% Do NOT change this "Section" title
% and do NOT add more "Section" level titles.
\section{Method}\label{sec:method}

During the beggining of this project a lot of resarch in control theory was made. The authors looked into PID control since that was the used technology in the VASA AUV. After further investigation the authors could conclude that PID control would suffice for this project.

The reason for this was the proven stability and usability of PID control.

In this section the used methodology in creating the motion control system for the Naiad AUV will be explained. During the start of the development of the component in question the authors looked into the current state of control theory research and found that the most widely used system is a PID controller based systems.

A proportional-integral-derivative controller (PID controller) is a generic control loop widely used in industrial control systems. The reason for its wide usage is the simplicity of the systems theory and implementation.

A PID controller is a self stabilizing controller that sets itself at a wanted value, with proper calibration of course. 

A PID controllers feedback system is based upon the current difference between the current output value of the system to the wanted output value of the system. 








