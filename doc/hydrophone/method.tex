% Do NOT change this "Section" title
% and do NOT add more "Section" level titles.
\section{Method}\label{sec:method}
The method used for finding the pinger location is called multilateration. It uses the difference in time at which the signals arrive at each hydrophone to localize the sound source. With the use of 2 hydrophones the location is narrowed down to a hyperboloid. One additional hydrophone gives an extra hyperboloid and the intersection of the two hyperboloids helps narrow down the location of the pinger to a curve. With one more hydrophone, another unique curve is obtained and the intersection of the two curves gives the exact location of the sound source. The x,y and z co-ordinates of the pinger are calculated considering one of the hydrophones as the origin.\newline
The derivation for the formulas was done first for the simplest case of hydrophones array, that is the hydrophones are arranged in a square. However due to the design constraints of the AUV, the hydrophones could only be placed in a rectangle. Hence the formulas were derived for the x,y and z co-ordinates of the sound source. After that the formulas were simplified so as to avoid float errors. Hence all the quantities are calculated as numerator and denominator and divided in the end.\newline
The inputs to the formulas are:\begin{enumerate}
\item Length and width of the rectangular array of hydrophones.
\item Velocity of sound in water.
\item Time of arrival of signal at each hydrophone.
\end{enumerate}
%Text, test citation \cite{web:website}.

%In chaper \ref{sec:introduction}, this topic was first introduced (Test of
%cross-reference with label).

% You can use how many "subsections" and "subsubsections" you like.
%\subsection{Subsection}
%Text, test citation \cite{article:article}.
%\subsubsection{Subsubsection1}
%Text, test citation \cite{unpublished:unpublished}.
%\subsubsection{Subsubsection2}
%Text, test citation \cite{book:book}.
