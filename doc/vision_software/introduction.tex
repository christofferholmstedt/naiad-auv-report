% Do NOT change this "Section" title
% and do NOT add more "Section" level titles.
\section{Introduction}\label{sec:introduction}
This report covers the work of the vision system software team of Naiad [4], which was the project undertaken as part of the course Project in Advanced Embedded Systems,[3] DVA425, at Mälardalens Högskola [1] in the fall semester of 2013. The team consisted of two international students in the final year of the Intelligent Embedded Systems program at the university.
The purpose of this team was to develop a vision system for an autonomous underwater vehicle, AUV, capable of providing real-time object detection, recognition, and velocity estimation. The system should also be capable of recreating a three dimensional map of an object using stereo vision.

\subsection{Problems faced}
One of the challenges faced by all underwater vision systems is the ever-changing lighting conditions under water. Rays of light shining down through the water can affect the image detection system. A system was needed to deal with this light disturbance.

vision underwater
-changing light conditions
-light diffusion
-light diffraction
-impaired visibility due to particles in water


\subsection{Software/Hardware}
There were two hard requirements of the project for the vision system. The first was that Ada should be the main programming language. This was to introduce safety critical features to the code and make the system very robust and safety critical. The other requirement was that the software should be developed for the research platform GIMME-2 [6]. 
For the real time image capture and perception OpenCV[5] was chosen and interfaced with the GIMME-2 and ADA. For the three Dimensional reconstruction Point Cloud Library [2], PCL, was chosen and also interfaced to work through Ada. (will be adding gimme2 here when working). The operating system chosen to develop the software was Ubuntu 12.04 [7]

The Choices for software/components and reasons:
1.GIMME2
    It was a hard requirement from the customer.
2. Ada
    It was a hard requirement from the customer.
3. OpenCV
    It seemed to be the obvious choice after a research was done on the state of the art technologies, and it was found that this library provides easy access to quite a few of the methods intended to be used in the system.
4. PCL
This library provides easy access to quite a few of the methods intended to be used in the system


OpenCV with Ada
Interfacing openCV with Ada would be a problem. Extensive research did not yield any instances of the current version of OpenCV (2.4.7) having support for ada. Bindings were found for an older version of OpenCV (2.0???) [8] but as OpenCV has undergone significant changes since then, most notably the method in which image data is handled, these could not be used.

PCL with Ada
As point cloud library is a relatively new software, 2011, [9] its support of different languages is limited. Much like OpenCV a method to incorporate pcl to ada was needed.

Hardware Platform
The state of the art hardware platform used for the project, the GIMME-2, presented several challenges to the software team. As the platform was state of the art, support for developing software for it was limited.


% You can use how many "subsections" and "subsubsections" you like.
\subsection{Subsection}
Text. This is a reference to table \ref{table:one_column}.
\subsubsection{Subsubsection1}
Text. This is a reference to table \ref{table:two_columns}.
