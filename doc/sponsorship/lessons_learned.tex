% Do NOT change this "Section" title
% and do NOT add more "Section" level titles.
\section{Lessons learned}\label{sec:lessons_learned}
During the course of this project some important lessons have been learned that I would like to share with whoever is thinking about setting up and taking on a role like this in coming year's project. See the following lessons as advices that will make your job as head of external relations easier and more successful, but also take them into consideration when deciding whether to dedicate a person to a role like this in your project or not.

\subsection{Contacting companies both is and is not calling a number on your contacts list}
If you thought that contacting companies is as quick as dialing/mailing a friend on your contacts list, then you are in for a big surprise. You have to get contact information to the right persons in the right positions in the right companies and this takes a lot more time than you can imagine. To figure out who they are and to get to them, you have to use your own contacts. If you are trying to just google for some company and send an email to their info-mail, then you are screwed. You will not get any answers. Even more annoying is that even if you know the name of somebody you believe is a good person to contact, then you most probably will not be able to find their contact information anywhere, as companies do not want you to contact their people directly in this way. If you are lucky you can figure out the company's e-mail address convention with some detective work, though. 

However, the most effective way to get in contact with the right people is to ask someone you know if they know someone that they believe is the right person to contact. Then you should refer to the common contact while mailing/calling that person. This will get you a higher answering-rate. Believe me when I say that you really need it.

\subsection{Do not lose hope, ever heard about the ketchup effect?}
You probably will start the project with high hopes that people will answer you, because that is what you are used to. If you call or mail someone in ordinary life, then they will answer. You might think that companies at least will respond to tell you that they are not interested. Well, if so, I am sorry to disappoint you by saying that is not the case.

However, do not lose hope! When it felt like no one was interested in the project, something happened and all the work was starting to pay off. All of a sudden we had got several sponsors and been on television and in several newspapers. When it seems as darkest you have to work the hardest.

\subsection{Companies are slowly moving machines}
So when you finally have a company willing to sponsor the project one might think that you will have the money on your account within a few working days. Unfortunately, this is very unlikely to happen. My experience says that this process most likely will take several weeks, even several months. So be prepared for this. Sometimes you need to be somewhat pushy in order to get companies to get their thumbs out. I would like to mention that you should not be afraid to tell them to hurry up. They will not back out of the deal, because they have more to lose than to win by doing so and they actually really want to sponsor you.

\subsection{Be prepared, honest, cocky and proud}
Sometimes one can feel somewhat little when having meetings with company leaders, asking them for money and they start asking questions and putting you against the wall. But do not get afraid by this situation - feel the heat, love it and seal the deal! To do this you have to be prepared. You have to know where the project is heading, what parts you would like to acquire, your budget and so on. If there is something you cannot answer, tell them that you will look it up and get back to them. Then you just sell the project to them by being proud of the work you have done/are doing and somewhat cocky.  Think positively! If they have taken time to meet you they are already interested. You have the advantage!

\subsection{Help is rare}
Sometime during the project, probably sooner than later, you will start to realize that everyone sort of just cares about their own work. If you ask for help people might tell you: "no, that's your work". This is not a great thing when, for instance,  working with marketing or when spreading a crowdfunding campaign. There is no way you will be able to get the spreading you need on your own. So you will have to work your ass off just to get the other project members to realize that they also have to spend some time helping you at different stages throughout the project. 

Basically people have to see the project as a distributed system. Tasks can be spread out and it is important that they all work individually, but if the different parts do not work together, then the complete system will never function well. See the bigger picture! 

To prevent this situation it might be a good idea to select a sponsoring and marketing team, so that more people feel that it is their responsibility to get more money to the project. Having to gather at least 200 000 kr it is crucial that you get this part of the project spinning on all wheels. Maybe people that are studying marketing can get involved with the project in some of their courses. 

\subsection{When you read this you should already have started}
As time has flown by I have realized that the work of gathering sponsors should have been started earlier in order to have a chance of getting the amount of money you need. Saying this, I started the work as early as the second week into the project. 

You will be limited by your small starting funds, not being able to buy the parts you need or would like to buy. This will slow down the progress of the project. So, see if there is a way to start this work earlier in some way. I would recommend some months before the project actually starts, even though I doubt that is possible. Maybe it could be done in cooperation with or by marketing students.

\subsection{Two roles is one to many}
Being both the software manager and head of external relations is not a great idea. This has been the case during this project and should be avoided. The workload in both managing roles are too heavy for one person.