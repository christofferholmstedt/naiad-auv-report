
\section{Conclusion}\label{sec:conclusion}
As mentioned in Section ~\ref{sec:result}, the implementation of the BeagleBone Black to CAN communication has worked well. However, there are ways the current design can be improved.

\subsection{Future work}
As mentioned in the previous sections, the protocol for the communication over UART is based on the sequence of bytes being correct at any time. One could implement some form of "start of message" and "end of message" flags to signal the start and end of messages. This would ensure that if one message is sent incorrectly, with too many or too few bytes, the receiver would be able to find the "start of message" and "end of message" flags and regain synchronization. \newline
Since any byte value can be sent in the payloads of the CAN messages, there exists no pattern of bytes that can be assumed not to occur in the data that will be sent over UART. Therefore some form of byte stuffing has to be implemented in order to handle this.

To improve the communication speed, the UART communication could be done via SPI instead. This would require rewriting some of the AT90CAN\_Uart\_To\_Can code as well as implementing SPI on the BBB.

The protocol could be be optimized further by only sending the message ID in two bytes instead of four if the message ID is not extended.

Tools like UPPAAL could be used to prove correctness and/or simulate the BBB to CAN subsystem and other nodes on the CAN bus.
