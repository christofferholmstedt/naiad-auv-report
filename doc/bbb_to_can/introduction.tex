
\section{Introduction}\label{sec:introduction}
Since all higher level computations is run on BeagleBone Black (abbreviated "BBB") boards, theses will have to be connected to the CAN-bus of Naiad. Research in the beginning of the project found that there was a \emph{Cape} (an extension board to the BBB) that could connect a BBB to CAN. However, further investigation revealed that this cape had issues and may not work. Because of this it was decided that each BeagleBone Black would be connected to the CAN bus via a Generic CAN controller. \newline
There were two possible ways to connect the BBB to the Generic CAN controller: via UART or SPI. UART was chosen for two reasons: In the Vasa project, a protocol for sending CAN messages over UART was already implemented for an AT90CAN128 microcontroller (the same that is used on the the Generic CAN controller) and UART allows the communication in one direction to be truly independent of the other direction. Using SPI would most definitely be possible but would require the communication protocol between the BBB and the Generic CAN controller to be more complicated.  \newline
The protocol used from the Vasa project was modified to better fit the requirements of the Naiad project.

Whenever the BBB needs to send a CAN message, it is converted into a string of bytes that will be sent over UART to the Generic CAN controller that would convert this string of bytes back to a CAN message and output it on the CAN bus. When the Generic CAN controller receives a CAN message it the same process will be done but in the opposite direction. 

The highest baud rate that was possible to achieve between the BBB and the AT90CAN128 microcontroller was 115200 baud per second (both BBB and the AT90CAN128 microcontroller can send at higher speeds but 115200 baud per second was the highest that both of them could run at). 