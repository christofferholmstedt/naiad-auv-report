
\section{Implementation}\label{sec:implementation}
The implementation has the following parts:

\begin{itemize}
   \item { \em The UART drivers for the AT90CAN128 microcontroller} \newline This part is described in the AT90CAN drivers report.
   \item { \em The CAN drivers for the AT90CAN128} \newline This part is described in the AT90CAN drivers report.
   \item { \em The UART drivers for BeagleBone Black} \newline This part is described in the BeagleBone Black drivers report.
   \item { \em Conversion between CAN message and strings of bytes } \newline This is implemented in the Can\_Utils project.
   \item { \em CAN interface code for the BeagleBone Black} \newline This is implemented in the BBB\_Can project.
   \item { \em The main program on the AT90CAN128} \newline This is implemented in the AT90CAN\_Uart\_To\_Can project.
\end{itemize}

The hardware filters in the CAN AT90CAN128 microcontroller are set to receive all CAN messages sent on the CAN bus, i.e. no filtering will be used. The reasoning for this is the following; 
If not all messages would be forwarded to the BBB, then the information about which messages that are to be forwarded would need to sent from the BBB to the Generic CAN controller. This could most definitely be done but it would make the communication protocol much more complicated.

\subsection{Protocol}
The highest baud rate that was possible to achieve between the BBB and the AT90CAN128 microcontroller was 115200 baud. Both BBB and the AT90CAN128 microcontroller can send at higher speeds but 115200 baud was the highest baud rate at which both of them could run. This relatively low baud rate (compared to the 250 kBaud used for the CAN bus, though the CAN protocol has a significant amount of overhead) means that the BBB to CAN link is the "bottle neck" of the whole CAN bus system. For this reason the protocol for sending CAN messages over UART was made as simple as possible to maximize performance, i.e. number of CAN messages that can be sent per second.

There are mainly four pieces of information in each CAN message: 
\begin{itemize}
   \item { Whether the message uses extended (29 bits) or normal (11 bits) message IDs} 
   \item { The length (number of bytes) of the payload} 
   \item { The message ID}
   \item { The payload of the message}
\end{itemize}

Each message is encoded in the form of a header of five bytes followed by the payload of zero to eight bytes. The first byte of the header contains information regarding whether or not the message uses an extended message ID and the length of the message. Its four least significant bits contain the length of the message. The fifth least significant bit is set to 1 if the message is extended and 0 if it is not.  \newline
The following four bytes of the header represent the message ID. \newline
After the header, the payload bytes (if any) follow. \newline
No "start of message" or "end of message" flags are used, to minimize the overhead even further.

\emph{Please note:} This protocol is dependent on the byte sequence between the two devices being precisely correct. Even one extra, one less, or (in some cases) a corrupted byte will cause an offset in the receiver buffer, and the two devices will become unsynchronized and the whole communication will be compromised. The same will happen if one or several bytes are lost e.g. due to a receive buffer overflowing. \newline

Since conversions between CAN messages and bytes would be done on many different devices (the BeagleBone Black, the Generic CAN controller and also in the Simulator), this functionality was put into the \emph{Can\_Utils} project, in order to be shared. \newline
The Can\_Utils project has three functions: Message\_To\_Bytes, Bytes\_To\_Message\_Header and Bytes\_To\_Message\_Data.  \newline
When converting bytes to a CAN message, the length of the message is unknown and therefore the conversion has to be done in two steps. 
In the first step, the bytes of the header (which is of constant length) are read and converted to the message header, from which the length of the payload data can be retrieved. In the second step, as many bytes as the payload size are read and converted to message data. \newline
This is the reason why two separate conversion procedures are provided: Bytes\_To\_Message\_Header and \newline Bytes\_To\_Message\_Data.  \newline
When converting a CAN message to bytes, the length of the message is known, and consequently the whole conversion can be done in one procedure, Message\_To\_Bytes.

\subsection{Interface code for the BeagleBone Black}
The BBB\_Can project uses the UartWrapper to send and receive data on the UART. BBB\_Can provides three procedures: \emph{Init}, \emph{Send} and \emph{Get}. \newline
\emph{Init} simply initiates the UartWrapper. \newline
The \emph{Send} procedure takes a CAN message as an argument, calls the \emph{Message\_To\_Bytes} procedure in \emph{CAN\_Utils} to get a string of bytes representing the CAN message and writes these bytes to the UART send buffer using UartWrapper. \newline
The \emph{Get} function has three out parameters: \emph{msg} (the CAN message received), \emph{bMsgReceived} (a boolean value set true if a CAN message was received) and \emph{bUARTChecksumOK} (which is obsolete and should not be read). 

The \emph{Get} procedure reads all the data in the UART receive buffer and puts it in a separate software buffer. The reason for this separate software buffer is that the UartWrapper does not provide a function for getting the number of bytes in the receive buffer without reading all available bytes. \newline
If there is enough data in the software buffer, the header of the message is read and converted to a CAN message. Given this, the number of bytes in the message payload is known and the remaining bytes are read from the software buffer. If not all payload bytes have been put into the software buffer, more data is read from the UART receive buffer until enough bytes can be read. The procedure will then return with the received CAN message and \emph{bMsgReceived} set true. \newline
If not enough data for a full header has been received, the procedure will return with \emph{bMsgReceived} set false, \emph{msg} will then be undefined and shall not be read.


\subsection{Main program running on the Generic CAN controller}
The main program running on the Generic CAN controller is called AT90CAN\_Uart\_To\_Can and has two main procedures:

\begin{itemize}
   \item { \em Handle\_Uart}
   reads CAN messages from the UART receive buffer and writes them to the CAN send buffer.
   \item { \em Handle\_Can}
   reads CAN messages from the CAN receive buffer and writes them on to the UART send buffer.
\end{itemize}

Since this program is single threaded it will alternate between these procedures. \newline 
As mentioned, the protocol for the communication over UART is sensitive to an overflow in the receive buffer. For this reason, the Handle\_Uart procedure has been prioritized and will handle all CAN messages in the buffer and only return when no more messages have been received. Handle\_Can will only handle one message, if any message is present in the CAN receive buffer, and then return. \newline
This way any overflow in BBB to CAN subsystem will only be in one of the following:

\begin{itemize}
   \item { \em The send buffer of the BBB} \newline
   If the BBB writes to its send buffer at a higher rate than the UART connection can send.
   \item { \em The CAN send buffer} \newline
   If the main program tries to send messages at a higher rate than it can send, for example during to high CAN bus utilization.
   \item { \em The CAN receive buffer} \newline
   If CAN messages are received from the CAN bus at a higher rate than what the main program can read. For example if the BBB is sending CAN messages at a high rate the Handle\_Can will not get enough time to empty the CAN receive buffer.
\end{itemize}

The first case is easily avoided by making sure the BBB does not send CAN messages at a faster rate than the BBB to CAN subsystem is capable sending. The second and third cases will only cause individual CAN messages to be lost rather than compromising the communication protocol between the BBB and the Generic CAN controller. \newline
This way the BBB to CAN communication subsystem can be considered fail-safe.