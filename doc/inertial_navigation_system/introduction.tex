\section{Introduction}\label{sec:introduction}
Any Autonomous Underwater Vehicle (AUV) such as the Naiad AUV is highly dependent on accurate information about its orientation and position in order to perform its missions. The inertial measurement unit (abbreviated "IMU" in this report) used in Project Vasa <INSERT REFERENCE> did not give sufficiently accurate orientation data. The main problem was the yaw angle (also known as the heading). \newline
The pitch and roll angles rarely deviate more 30 degrees during the course of a normal AUV mission (for either the Vasa or Naiad AUVs). The yaw angle however, can turn several full rotations during the course of a mission. Moreover, the pitch and roll angles can easily be corrected using the gravity vector as a reference. This cannot be done with the yaw angle since the yaw angle moves in a plane perpendicular to the gravity vector. \newline
For this reason many IMUs, including that used on the Vasa AUV, rely on a magnetometer that measures the earth's magnetic field to correct the yaw angle. This may work well, but in some cases the magnetometer can be interfered by other electromagnetic fields created by equipment such as electric motors (including those on the vehicle itself), generators, power lines, etc. For this reason it was decided that the Naiad AUV was to be equipped with a fiber optic gyroscope (abbreviated "FOG" in this report) in order to get a separate sensor for the yaw angle.

All the electric circuitry needed to interface with the IMU and the FOG was put on one circuit board known as the \emph{INS board}. This board is "stacked" on top of a Generic CAN controller. The \emph{Generic CAN controller} is a small (approx. 76 by 40 millimetres) electronic board that has an AT90CAN128 microcontroller, an MCP2551 CAN transceiver and all the peripheral circuitry needed for these as well as its own power supply. The Generic CAN controller is used throughout the project for several tasks. It can be connected to the CAN bus and has two UART buses as well as an SPI bus. \newline
The INS board is powered by the Generic CAN controller and provides it with the SPI output from the FOG and a UART interface with the IMU. The Generic CAN controller sends the readings of the IMU and FOG the CAN-bus to Sensor fusion board.
