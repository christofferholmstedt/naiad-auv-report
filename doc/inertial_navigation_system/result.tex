
\section{Result}\label{sec:result}
The majority of the inertial navigation system (INS) was implemented during the project. However, there are some bugs that cause the AT90CAN\_Ins\_Controller software to crash. At the time of writing, it is believed that this is caused by some form of error that occurs when the micro controller is programmed. Unfortunately, there has not been enough time to prioritize these problems. Documentation has been seen as more important.

The FOG was never connected to the INS board. Since the FOG is an expensive and sensitive piece of equipment it was decided only to connect it once all the hardware and firmware was done and meaningful tests could be performed. This point was not reached during the project.

The IMU was tested during the project and it works very well. Only a few simple experiments were performed in order to test the drift of the orientation readings of the IMU. These tests were only performed during short durations (minutes), so one should not put too much trust in them. The results seemed to show that the IMU performs very well with very little drift and that it is almost immune to magnetic disturbances.

Throughout all the testing that has been done during the project, no cases of data corruption (bit flips etc.) of commands sent to the IMU or data received from the IMU have been experienced. Hence, the assumption that the checksums were not needed seems to be correct.
