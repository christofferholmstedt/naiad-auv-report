% THIS IS SIGPROC-SP.TEX - VERSION 3.1
% WORKS WITH V3.2SP OF ACM_PROC_ARTICLE-SP.CLS
% APRIL 2009
%
% It is an example file showing how to use the 'acm_proc_article-sp.cls' V3.2SP
% LaTeX2e document class file for Conference Proceedings submissions.
% ----------------------------------------------------------------------------------------------------------------
% This .tex file (and associated .cls V3.2SP) *DOES NOT* produce:
%       1) The Permission Statement
%       2) The Conference (location) Info information
%       3) The Copyright Line with ACM data
%       4) Page numbering
% ---------------------------------------------------------------------------------------------------------------
% It is an example which *does* use the .bib file (from which the .bbl file
% is produced).
% REMEMBER HOWEVER: After having produced the .bbl file,
% and prior to final submission,
% you need to 'insert'  your .bbl file into your source .tex file so as to provide
% ONE 'self-contained' source file.
%
% Questions regarding SIGS should be sent to
% Adrienne Griscti ---> griscti@acm.org
%
% Questions/suggestions regarding the guidelines, .tex and .cls files, etc. to
% Gerald Murray ---> murray@hq.acm.org
%
% For tracking purposes - this is V3.1SP - APRIL 2009

\documentclass{acm_proc_article-sp}

% For proper urls in the reference list
\usepackage[hyphens]{url}

\begin{document}

\title{Inertial navigation system\titlenote{This report was
        written during the fall of 2013 in an advanced level project course
        at M\"{a}lardalen University, Sweden.}}
%\subtitle{[Extended Abstract]
%\titlenote{A full version of this paper is available as
%\textit{Author's Guide to Preparing ACM SIG Proceedings Using
%\LaTeX$2_\epsilon$\ and BibTeX} at
%\texttt{www.acm.org/eaddress.htm}}}
%
% You need the command \numberofauthors to handle the 'placement
% and alignment' of the authors beneath the title.
%
% For aesthetic reasons, we recommend 'three authors at a time'
% i.e. three 'name/affiliation blocks' be placed beneath the title.
%
% NOTE: You are NOT restricted in how many 'rows' of
% "name/affiliations" may appear. We just ask that you restrict
% the number of 'columns' to three.
%
% Because of the available 'opening page real-estate'
% we ask you to refrain from putting more than six authors
% (two rows with three columns) beneath the article title.
% More than six makes the first-page appear very cluttered indeed.
%
% Use the \alignauthor commands to handle the names
% and affiliations for an 'aesthetic maximum' of six authors.
% Add names, affiliations, addresses for
% the seventh etc. author(s) as the argument for the
% \additionalauthors command.
% These 'additional authors' will be output/set for you
% without further effort on your part as the last section in
% the body of your article BEFORE References or any Appendices.

\numberofauthors{2}
% I've updated the number of authers ~ Christoffer 2012-10-14

%  in this sample file, there are a *total*
% of EIGHT authors. SIX appear on the 'first-page' (for formatting
% reasons) and the remaining two appear in the \additionalauthors section.
%
\author{
% You can go ahead and credit any number of authors here,
% e.g. one 'row of three' or two rows (consisting of one row of three
% and a second row of one, two or three).
%
% The command \alignauthor (no curly braces needed) should
% precede each author name, affiliation/snail-mail address and
% e-mail address. Additionally, tag each line of
% affiliation/address with \affaddr, and tag the
% e-mail address with \email.
%
% 1st. author
\alignauthor
Nils Brynedal Ignell\\
       \email{nbl08001@student.mdh.se}
% 2nd. author
\alignauthor
Aseem Rastogi \\
       \email{aar10001@student.mdh.se}
}
% There's nothing stopping you putting the seventh, eighth, etc.
% author on the opening page (as the 'third row') but we ask,
% for aesthetic reasons that you place these 'additional authors'
% in the \additional authors block, viz.
% \additionalauthors{Additional authors: John Smith (The Th{\o}rv{\"a}ld Group,
%email: {\texttt{jsmith@affiliation.org}}) and Julius P.~Kumquat
%(The Kumquat Consortium, email: {\texttt{jpkumquat@consortium.net}}).}
%\date{30 July 1999}
% Just remember to make sure that the TOTAL number of authors
% is the number that will appear on the first page PLUS the
% number that will appear in the \additionalauthors section.

\maketitle
\begin{abstract}
Naiad is capable of interacting with the environment using sensors and actuators. One of the short term goals of the prototype is to compete in the RoboSub, thus the current system is adapted according to the competition's mission requirements; there are two torpedo launchers, two markers, and two grippers. All the actuators are pneumatic-operated and are driven by separate and independent valves. The pneumatics system is controller by a Generic Can Controller, running the firmware written for this task.
\end{abstract}

% A category with the (minimum) three required fields
% \category{A.1}{General Literature}{Introductory and Survery}
%A category including the fourth, optional field follows...
% If we want to add another category (or several).
%\category{D.2.8}{CHANGE THIS Software Engineering}{Metrics}[complexity measures, performance measures]

% \terms{Theory}

% \keywords{Threat Models, Centralised systems, Decentralised systems, Censorship, Privacy, Natural disasters} % NOT required for Proceedings

\section{Introduction}\label{sec:introduction}
Any Autonomous Underwater Vehicle (AUV) such as the Naiad AUV is highly dependent on accurate information about its orientation and position in order to perform its missions. The Inertial Measurement Unit (abbreviated "IMU" in this report) used in Project Vasa~\cite{unpublished:vasa} did not give sufficiently accurate orientation data. The main problem was the yaw angle (also known as the heading). \newline
The pitch and roll angles rarely deviate more 30 degrees during the course of a normal AUV mission (for either the Vasa or Naiad AUVs). The yaw angle however, can turn several full rotations during the course of a mission. This causes the yaw angle to drift significantly more than the pitch and roll angles. Moreover, the pitch and roll angles can easily be corrected using the gravity vector as a reference. This cannot be done with the yaw angle since the yaw angle moves in a plane perpendicular to the gravity vector. \newline
For this reason many IMUs, including that used on the Vasa AUV, rely on a magnetometer that measures the Earth's magnetic field to correct the yaw angle. This may work well, but in some cases the magnetometer can be subject to interference from other electromagnetic fields created by equipment such as electric motors (including those on the AUV itself), generators, power lines, etc. For this reason it was decided that the Naiad AUV was to be equipped with a Fiber Optic Gyroscope (abbreviated "FOG" in this report) in order to get a separate sensor for the yaw angle.

All the electric circuitry needed to interface with the IMU and the FOG was put on one circuit board known as the \emph{INS board}. This board is "stacked" on top of a Generic CAN controller. The \emph{Generic CAN controller} is a small (approx. 76 by 40 millimetres) electronic board that has an AT90CAN128 microcontroller~\cite{web:at90can}, an MCP2551 CAN transceiver~\cite{web:mcp2551} and all the peripheral circuitry needed for these as well as its own power supply. The Generic CAN controller is used throughout the project for several tasks. It can be connected to the CAN bus and has two UART buses as well as an SPI bus. \newline
The INS board is powered by the Generic CAN controller and provides it with the SPI output from the FOG and a UART interface with the IMU. The Generic CAN controller sends the readings of the IMU and FOG over the CAN bus to the Sensor fusion board.

%% Do NOT change this "Section" title
% and do NOT add more "Section" level titles.
\section{Method}\label{sec:method}
The method used for finding the pinger location is called multilateration. As described in \cite{web:multilateration}, it uses the difference in time at which the signals arrive at each hydrophone to localize the sound source. With the use of 2 hydrophones the location is narrowed down to a hyperboloid. One additional hydrophone gives an extra hyperboloid and the intersection of the two hyperboloids helps narrow down the location of the pinger to a curve. With one more hydrophone, another unique curve is obtained and the intersection of the two curves gives the estimate location of the sound source. The x-,y- and z- coordinates of the pinger are calculated considering one of the hydrophones as the origin.\newline
With \cite{article:multilateration} as guideline and starting reference, the formulas were derived firstly for the simplest case of hydrophones arranged in a square array. However due to the design constraints of the AUV, the hydrophones could only be placed in a rectangle. Hence the formulas were derived for the x,y and z co-ordinates of the sound source. Afterwords the formulas were simplified so as to avoid float errors. Hence all the quantities are calculated as numerator and denominator and divided in the end.\newline
The inputs to the formulas are:\begin{enumerate}
\item Length and width of the rectangular array of hydrophones.
\item Velocity of sound in water.
\item Time of arrival of signal at each hydrophone.
\end{enumerate}
%Text, test citation \cite{web:website}.

%In chaper \ref{sec:introduction}, this topic was first introduced (Test of
%cross-reference with label).

% You can use how many "subsections" and "subsubsections" you like.
%\subsection{Subsection}
%Text, test citation \cite{article:article}.
%\subsubsection{Subsubsection1}
%Text, test citation \cite{unpublished:unpublished}.
%\subsubsection{Subsubsection2}
%Text, test citation \cite{book:book}.


\section{Implementation}\label{sec:implementation}
The temperature sensor used was the \emph{DS18B20} digital temperature sensor~\cite{web:ds18b20}. This sensor communicates over a 1-Wire interface. Since the Generic CAN controller does not have a 1-Wire interface, a DS2480B integrated circuit was used to convert between the 1-wire interface and UART (which the Generic CAN controller supports).

The pressure sensor used was the \emph{IMCL Low Cost Submersible Pressure Sensor}~\cite{web:imcl}. The interface from the pressure sensor could be chosen when ordering the sensor so a simple analog voltage in the 0 to +5 Volt range was chosen since this is the range of the analog-to-digital converter on the AT90CAN128 microcontroller. Consequently, no interface circuitry was needed for the pressure sensor.

The salinity sensor chosen was the \emph{Atlas Scientific E.C Circuit}~\cite{web:ec_circuit}. The E.C Circuit is connected to a probe whose output the E.C Circuit measures. The E.C Circuit has a serial interface that is connected to one of the UART busses of the AT90CAN128 microcontroller. No level conversion is needed since the E.C Circuit's serial interface uses +5 Volt TTL levels.

The E.C Circuit outputs the conductivity of the water, the Total Dissolved Solids (TDS) of salt in the water, as well as the salinity expressed as an integer number in the Practical Salinity scale of 1978. Out of these, only the last is used. \newline
Three types of probes can be used. One for fresh water, one for brackish water and one for salt water. The ranges of salinity each probe can measure do overlap, but a correct probe must be chosen or the salinity reading can come out of range. 

Readings from all sensors will be put in the same CAN message containing 5 payload bytes according to Table~\ref{table:SensorMessage}. This CAN message is called the \emph{Sensor\_Message} and is sent at regular intervals.

\begin{table*}
\centering
    \caption{Payload bytes of the Sensor CAN message}
    \begin{tabular}{|l|l|p{11cm}|} \hline
    \label{table:SensorMessage}
    	\textbf{Bytes} & \textbf{Data type} & \textbf{Meaning} \\ \hline
        1 - 2 & Integer\_16 & Temperature in 16\textsuperscript{ths} of degrees C, e.g. +20.25 will be sent as 324 \newline 
        (00000001 01000100 in binary). \\ \hline
        3 - 4 & Unsigned\_16 & Pressure in millibars. \\ \hline
        5  & Unsigned\_8 & Salinity in the Practical salinity scale, 0-42. 255 means that the value is out of range. 254 means no measurement has yet been done. \\ \hline
    \end{tabular}
\end{table*}

\subsection{Temperature sensor}
To receive the data from the temperature sensor a protocol converter between UART and 1-wire communications has been used. In order to read the temperature a number of characters have to be sent from the UART to the converter which interprets the received characters and generates the corresponding 1-wire signals. The temperature sensor detects the signals, makes an analog-to-digital conversion of the temperature and puts the data on the 1-Wire bus. Afterwards, the converter takes the data from the 1-Wire bus, translates it into characters and sends them over the UART to the AT90CAN128. 

On the AT90CAN128 the characters are computed into a value with a resolution of 1 degree Celsius, which is more than enough, since the speed of sound doesn't change that much at this little temperature variation. A procedure has been written so that when it is called it will take a temperature reading and return the computed temperature.

\subsection{Pressure sensor}
Since the pressure sensor is analog, the AVR.AT90CAN128.ADC library~\cite{web:naiad_git} is used to read the analog output of the sensor whenever a CAN message is to be sent with sensor readings. This analog-to-digital value is converted to millibars before being sent as a CAN message. 


\subsection{Salinity sensor}
The E.C Circuit~\cite{web:ec_circuit} is a circuit board in itself with header pins on its underside. These are in turn connected to matching header sockets on the Sensor Controller Board. \newline
The E.C Circuit is dependent on a temperature reading of the water in order to provide accurate readings. Hence, the AT90CAN\_Sensor\_Controller software first makes a temperature measurement before sending a command to the the E.C Circuit. The command will be of the following format:

\begin{quote}
17.5,C\emph{<carriage return>}
\end{quote}

The "17.5" is the measured temperature, the "C" tells the E.C Circuit to make continuous measurements. When the temperature changes, a new command is sent to the E.C Circuit in order to keep it updated with an accurate temperature reading. \newline
The responses from the E.C Circuit have the following format:

\begin{quote}
50000,32800,32\emph{<carriage return>}
\end{quote}

The last number ("32" in this example) represents the salinity in the Practical Salinity scale 1978. The two other numbers represent the conductivity of the water, the Total Dissolved Solids (TDS) of salt in the water and are discarded. The salinity reading will be in the range 0 to 42. If the reading was out of range the value "\--\--" will replace the numerical value. \newline
The salinity reading will be output from the E.C Circuit at regular intervals. The latest reading is stored as a variable so that it can easily be fetched when it is time to send a CAN message with sensor data. This variable is initiated at 254, meaning that no measurement has been done. If the salinity sensor is not connected to the UART bus of the Generic CAN controller, this variable will never be set to any other value and consequently the value 254 will always be output on the CAN bus.


% Do NOT change this "Section" title
% and do NOT add more "Section" level titles.
\section{Result}\label{sec:result}
Tests with the bootloader and configuration manager has been successful in programming AT90CAN microcontrollers by sending messages from the PC to the BeagleBone Black and then onwards onto the CAN-bus.

There has yet to be any test to see how it works together with the whole system because all the necessary hardware isn't available.

A sceenshot of the layout of the configuration manager can be found in Appendix \ref{sec:cm_appendix}.


% You can use how many "subsections" and "subsubsections" you like.
%\subsection{Subsection}
%After this comes a table or at least somewhere close by...perhaps in the top
%of the column?
%
%
%\subsubsection{Subsubsection1}
%After this comes a table or at least somewhere close by...perhaps in the top
%of the next page?



\section{Conclusion}\label{sec:conclusion}
Text

\subsection{Subsection}
Text
\subsubsection{Subsubsection1}
Text


%\end{document}  % This is where a 'short' article might terminate

% Just comment this out if we don't need it.
%\input{acknowledgment.tex}

%
% The following two commands are all you need in the
% initial runs of your .tex file to
% produce the bibliography for the citations in your paper.
\bibliographystyle{abbrv}
\bibliography{../references}  % references.bib is the file with all references. 
% You must have a proper ".bib" file
%  and remember to run:
% latex bibtex latex latex
% to resolve all references
%
% ACM needs 'a single self-contained file'!
%
%APPENDICES are optional
\balancecolumns
%\appendix
%% Each section is a new appendix
\section{Appendix}
\subsection{Getting started with GIMME-2 board}\label{sec:first_appendix}
This section describes the step by step procedure to get started with the GIMME-2 board. The operating system used on the host machine is Ubuntu 12.04. The prerequisites for initializing the GIMME-2 board are:\begin{enumerate}
\item Micro-USB Male (Type B) to USB Male (Type A) connector or an Ethernet cable.
\item Power supply. The board should be supplied with voltages in the range +12V to +24V.
\item If the user is planning to use the USB cable, then a serial terminal, for example \textit{gtkterm}, should be installed on the host PC to interact with the Linux on the GIMME-2 board.
\end{enumerate}

Assuming that Zynq Linux is already installed on the GIMME-2 board, the user can interact with it in either of the two ways described below. Ensure that the jumper configuration is 0100 before powering on the board.
\subsubsection{Using serial interface}
Connect the micro-USB to USB connector to the port labelled \textit{USB0} on the board and connect the other end to the host PC. Set the voltage of the power supply to +12V and connect it to the GIMME-2 board. Now open the serial terminal and the booting steps can be seen there. Once the booting is over, the user gets a \texttt{zynq>} prompt on the screen.

\subsubsection{Using Ethernet}
Connect one end of the Ethernet cable to the middle port among the three on the GIMME-2 board and the other end to the host machine. Now open the bash terminal and execute
\texttt{ssh root@192.168.1.10}. Type in \textit{root} when password is prompted. The user gets a \texttt{zynq>} prompt on the screen.

\subsection{Creation of Zynq Boot Image}
The procedure for creating a boot image is quite complex and time consuming. For the sake of the new users, all the files required to create a boot image are available on the Subversion repository for project Naiad \cite{web:svnGimme}. Additional to the prerequisites mentioned in section 5.1, the user needs to do the following:\begin{enumerate}
\item Install the full package of Xilinx ISE Design Suite\cite{web:downloadISE} including the Software Development Kit (SDK) on the host PC.
\item Install the driver for Xilinx Platform USB Cable II \cite{web:driverCable}. If Step 1 is done in Windows OS, then no need to install the driver as it comes with the package.
\end{enumerate}

Using SDK, new boot image can be created by giving offsets to the binary files as mentioned in the user guide \cite{web:svnGimme}. If amendments have to be made to the Linux kernel like installing a new Linux driver, then the RAM disk image should be modified and a new boot image should be created with the new RAM image. All the steps for modifying RAM image are also detailed in the user guide \cite{svnGimme}.

%\section{Headings in Appendices}
%The rules about hierarchical headings discussed above for
%the body of the article are different in the appendices.
%In the \textbf{appendix} environment, the command
%\textbf{section} is used to
%indicate the start of each Appendix, with alphabetic order
%designation (i.e. the first is A, the second B, etc.) and
%%a title (if you include one).  So, if you need
%hierarchical structure
%\textit{within} an Appendix, start with \textbf{subsection} as the
%highest level. Here is an outline of the body of this
%document in Appendix-appropriate form:
%\subsection{Introduction}
%\subsection{The Body of the Paper}
%\subsubsection{Type Changes and  Special Characters}
%\subsubsection{Math Equations}
%\paragraph{Inline (In-text) Equations}
%\paragraph{Display Equations}
%\subsubsection{Citations}
%\subsubsection{Tables}
%\subsubsection{Figures}
%\subsubsection{Theorem-like Constructs}
%\subsubsection*{A Caveat for the \TeX\ Expert}
%\subsection{Conclusions}
%\subsection{Acknowledgments}
%\subsection{Additional Authors}
%This section is inserted by \LaTeX; you do not insert it.
%You just add the names and information in the
%\texttt{{\char'134}additionalauthors} command at the start
%of the document.
%\subsection{References}
%Generated by bibtex from your ~.bib file.  Run latex,
%then bibtex, then latex twice (to resolve references)
%to create the ~.bbl file.  Insert that ~.bbl file into
%the .tex source file and comment out
%the command \texttt{{\char'134}thebibliography}.
% This next section command marks the start of
% Appendix B, and does not continue the present hierarchy
%\section{More Help for the Hardy}
%The acm\_proc\_article-sp document class file itself is chock-full of succinct
%and helpful comments.  If you consider yourself a moderately
%experienced to expert user of \LaTeX, you may find reading
%it useful but please remember not to change it.
%\balancecolumns
% That's all folks!
\end{document}
