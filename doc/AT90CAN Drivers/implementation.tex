% Do NOT change this "Section" title
% and do NOT add more "Section" level titles.
\section{Implementation}\label{sec:implementation}
The following sections describe the drivers of various modules of AT90CAN used in Naiad project.
\subsection{CAN Drivers}
The AT90CAN consists of 1 CAN module. The CAN module has 15 Message Objects (MObs) from 0 to 14. Each of the MOb can be configured for either RX or TX. For the drivers in NAIAD, the MObs 0 to 12 are used for RX and 13 is used for TX. Out of the RX MObs only 0 to 9 can be configured by the user. For Naiad each board containing the AT90CAN can be in different modes like Bootloader mode, Simulation mode or Normal mode. The CAN module for different boards should operate differently in different mode, for example in Simulation mode the INS Board should not send any CAN messages. The Board\_And\_Mode\_Defs define how each board should react in each mode.The drivers primarily consist of following functions:
\begin{enumerate}
\item \textbf{Initialization (Can\_Init):} This is used to initialize the CAN module. It takes input the baud rate and Board name. The mode is initialized to Normal mode. The initialization sets the baud rate, clears all MObs and enables CAN. It also sets the MObs 10, 11, 12 to receive Bootloader start message, Mode message and Status request message respectively. These are the messages that all the boards should receive.
\item \textbf{Set Filter and Mask (Can\_Set\_MOb\_ID\_MASK) :} This is used to configure the RX MObs from 0 to 9. This take inputs the MOb number and the filter and mask. For AT90CAN the MOb filter and mask is cleared once a message is received in that register and the user has to re-configure it. To avoid that hassle the filter and mask for each MOb is stored in an array and the MOb is automatically reconfigured with the filter and mask once a message is received in it.
\item \textbf{Software Buffers :}The CAN drivers have RX and TX buffers of 16 messages each. The buffer is a ring buffer, and has pointers pointing to the read location and write location. The buffer pointers can have values from 0 to 31 even though the buffers are from 0 to 15. This is done to distinguish the full buffer from an empty buffer. When the buffer is empty both read and write will have same pointer location while in case of full they will differ by 16. In order to access the buffer, the buffer pointer mod 16 is used. 
\item \textbf{CAN Interrupt :}The CAN interrupt occurs whenever a message is transmitted or received in the MObs. The Interrupt iterates over all MObs and checks if a message is received or transmitted. \newline
For the receive interrupt it checks if the message is Mode message the mode is changed to the mode in the message, if the message is Reboot the Reboot function is called and if the message is to start the bootloader the boolean variable is set to True. Then the message is put in the message software buffer if the mode allows the CAN module to receive the messages. Otherwise the message is discarded. Then the RX Mobs filter and mast are reset by reading the values from the arrays.\newline
For the transmit interrupt, when the message has been transmitted the message is removed from the software buffer and the next message is put in the RX register for transmitting.
\item \textbf{Reading CAN Messages (Can\_Get) :}This is used to get the messages received. The function takes input the time it should wait for the message and returns a message and a boolean variable to tell if the message is received. A value of -1 can be passed in waiting time to make the funtion wait infinitely till a message arrives. The function iterates through all the messages in the software buffers and returns the highest priority message. Also the returned message is removed from the software buffer. This function is also calls the switch\_to\_bootloader message after the mode is set to bootloader and the mode message has been read by the user.
\item \textbf{Transmitting CAN Messages (Can\_Send) :}This function takes input the message to send over the CAN bus. If the current mode for that board allows for a message to be sent the message is put into the TX software buffer otherwise the message is discarded.
\end{enumerate}

\subsection{SPI Drivers}
The AT90CAN has 1 8 bit SPI module. The SPI drivers primarily has following functions:
\begin{enumerate}
\item \textbf{Initialization (Init) :}The Init function takes input the Clock divisor to use for the SPI clock, the clock mode, a boolean variable to tell if the device is master or slave, boolean to tell if the most significant bit is transmitted first and a boolean to tell if the Slave Select pin is used or not. The function disables the SPI device, configures it with the values passed and enables it.
\item \textbf{SPI transmit and receive :}There are three functions to transmit and receive SPI data depending on if only transmit, only receive or both transmit and receive is required. 
The SPI\_IO function takes input a byte which it transmits on the SPI and waits for the reply. It calls the master or slave IO functions depending on whether the AT90CAN is master or slave. The reply is then returned. \newline
The WriteSPI is used to transmit 1 byte of data. This function calls SPI\_IO internally. Whatever is received during that SPI transmit is discarded. \newline
The ReadSPI is used to read a byte of SPI data. This function also calls SPI\_IO internally. A temp value of 0 is transmitted and the received value is returned.

\end{enumerate}