
\section{Result}\label{sec:result}
Since the project ran out of time not all of the goals set for the Sensor Controller were achieved. 

\subsection{Temperature sensor}
A small PCB prototype which contains the converter circuit was built and connected to the temperature sensor and the Generic CAN controller. The tests results were very good. The entire temperature range was not tested, only from +15 degrees Celsius to +100 degrees Celsius due to the lack of equipment to create the rest of the temperature range, but for this range the sensor performed very well. 

\subsection{Pressure sensor}
The pressure sensor was never put through any major tests. This was due to time limits and the inherent difficulty of creating sufficient levels of pressure to perform a test. \\
The only practical way to do this is to submerge the sensor in water. The depth needed to test the whole range of pressure would mean that either a very long cable would need to be attached to the sensor (and made waterproof), or, that one puts the sensor, the Sensor controller and some hardware for  logging the sensor readings in a waterproof hull. The latter could be easily be done once the AUV's hull is completed, but this stage was not reached during the duration of the project.

A simple test was done where it was connected to a voltage source and its output voltage measured. The result from this test was good. However, more testing and calibration will be needed.

\subsection{Salinity sensor}
Since the salinity sensor was never purchased, no results regarding it were obtained.

