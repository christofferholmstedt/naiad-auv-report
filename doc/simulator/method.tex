% Do NOT change this "Section" title
% and do NOT add more "Section" level titles.
\section{Method}\label{sec:method}

Approaches used for solving problematic areas regarding the the implementation of the simulator, such a the overall structure of the program, simulated motion, representation and communication is explained in this section.


\subsection{Representation}

To represent the simulator, the choice of using GTK was obvious since the simulator was written in Ada. The reason for this was documentation, even though even though it was not greatly documented it was still far better than any of the other options that almost completely lacked documentation. The aspect of documentation was also combined with all the graphical tools that were needed to create the clean, informative and user friendly interface. Another aspect for making GTK Ada the better choice was that it was available on multiple platforms including Windows and Ubuntu.
\subsection{Design}

For the structure of the program between the Graphical interface, the motion calculation and the Ethernet sockets for communication with the AUV, the Model View ViewModel (MVVM) approach was decided upon. This is due to the clean coding with a clear interface between each layer in the program and a simple way of integrating the pieces. Here the top part which is the graphical interface calls for the underlying layers. This follows both the object oriented approach that is used for this project and implements well with Ada AUnit testing as well as there will not be any logic mixed up with the interface layer using that approach making parts easier to test.
\subsection{Motion Simulation}

Simulating the motion of the AUV is based on numeric integration of the accelerations that can be calculated using the functions described below. The method that will be used is to calculate the accelerations in a local coordinate system. The features such as the form of the AUV, motors and torques are already defined in the local coordinate system. This results in converting the velocities and angular velocities to the local frame using the inverse of the orientation matrix used to describe the orientation of the AUV. The local velocities are then used for the acceleration formulas as it affects both the friction force and the angular acceleration straight of, using formulas for general space manipulation described in the book Robotics, Vision and Control \cite{book:corke2013}.
\begin{equation}
Global Vector = Orientation*Local Vector
\end{equation}
\begin{equation}
Acceleration = \frac{\sum Force}{Mass}
\end{equation}
\begin{equation}
J*\alpha = - \omega \times J \cdot\omega + T 
\end{equation}


The moment of inertia and the mass of the AUV are both automatically calculated in the Solid Works CAD program used for designing the hull of the AUV. The affect of friction is less predictable and hard to mathematically calculate without expensive software. Assumptions regarding that a close to symmetric object actually is symmetric will be used to simplify the problem while still resulting in accurate approximations according to \cite{4772083}.

The accelerations are then converted back to the Global reference frame, allowing for the numerical integration of the velocities, positions, angular velocities and orientation. This results in the orientation being represented as a matrix and the other sets of data as vectors.
\subsection{Communication}

The communication of the simulator will be used to see the flow of data going between the different parts of the AUV, which is passed on by the sensor fusion module to the simulator. The simulator will send out messages when in simulation mode, where it simulates a sub system. This is done by receiving all inputs and sends corresponding outputs for that specific subsystem.

The communication between the simulator and the BeagleBone Black, which is what the simulator will be connected to in the AUV will be done over Ethernet. For communicating over Ethernet TCP sockets will be used as they are easy to use and give reliable performance. The set of data sent over the TCP will be following the same standard as the data sent over the CAN bus, which will make it simple to forward data by just pushing all data from CAN to Ethernet and vice versa without having to convert the messages.
