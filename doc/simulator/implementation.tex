% Do NOT change this "Section" title
% and do NOT add more "Section" level titles.
\section{Implementation}\label{sec:implementation}

% You can use how many "subsections" and "subsubsections" you like.
\subsection{Overall design MVVM}
The overall design of the simulator mainly following the MVVM standard was working as following. At the top of everything is the graphical interface of the user, which in turn holds several view models which it uses to perform its changes on. Each view model containing its own logic separated to that of the model itself. There are 7 view models that all work on different parts of the program, one for graphical representation one for the PID values one for the PID errors and one for setting each type of parameter.

These view models then shares one common model that they are all keeping access to. This model in turns holds some simple logic together with the motion simulator part of the simulator together with the different communication accesses, both with CAN over Ethernet and with a direct link to motion control. Where it with the update command fetches information of both the simulated motion part and the communication part and sends it to the other respective part.

\subsection{Usability and design in Gtkada}
The design of the user interface is built up as in picture [?] where one can see the 3 views showing positioning in the 2 respective axis. The interface is built up from using buttons where popup windows comes up for setting data or monitoring actuators such as the motors without bloating the screen when they are not an interesting part. To perform the actions with an as logic free graphical interface as possible the accessible view models are called for each action that is created on the graphical interface as well as a periodic update function used to updating positioning etc.

\subsection{Motion Simulation}
The implementation of the motion simulation is based on figuring out the torque and the forces of the motors. These are then used together with Inertia, frictions and buoyancy and other aspects stored in the simulated motion to calculate the acceleration and angular acceleration that they would affect the AUV with at that specific time step.
 
The acceleration and angular acceleration is then used with a numerical integration to receive the changes in velocity and angular velocity during that time frame, combining it with previous knowledge giving us velocity and angular velocity. The angular velocity and angular velocity are then used in the same way and integrated into position and orientation. 

\subsection{Communication}

The communication over TCP follows a simple structure for communication, two protected objects with the data that is wished to be sent over is stored in a protected object. From this protected object a sender task retrives the data it needs create the respective CAN messages and sends them.

On the other side of the communication package a receiver task reads the incoming CAN messages and writes the informative data into a receivers protected object which is read when the simulator requests data from the communication section.

\subsection{Unit tests with AUnit}

For testing of the sub parts of the system AUnit was used, it was used to check that each piece of logic was working as expected in order to full fill the overall performance requirements of the system.  All the possible correct outcomes were to be predicted so that each sub system could be guaranteed to work properly. Though in the end as time was a key issue it was implemented on all key parts of the system which was easily accessible for testing and had a quite high risk of failure. Priorities to not test all part of the simulator was done where risks of failure was not high as the functionality was simple and could be tested lightly in other ways. This ended up saving key time while still getting the overall results that were wanted from the simulator in time.

