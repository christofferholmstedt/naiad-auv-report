% Do NOT change this "Section" title
% and do NOT add more "Section" level titles.
\section{Introduction}\label{sec:introduction}
The simulator is a test platform for the AUV, which is connected to the AUV with full access to all information sent between the different sub systems of the AUV. The task of the simulator is that it should read all data sent from the AUV in order to either observe its movement or use the data for simulating a sub system of the AUV sending out data over the CAN bus. This allows the simulator to see the response of the other parts of the system, such as the wanted motor power and which actuators that would be used. The calls to the actuators are then being observed in order to make the same actions in the simulator to feed the AUV with simulated sensor data.

\subsection{Observation}
When the simulator is observing the AUV regarding how it behaves in water with regards to position orientation. At this point the data will not only be showed on the GUI but will also be used as vital information that can be used in the future for improve the simulators performance.

\subsection{Simulation at a pace faster than real-time}
The final task of the simulator is to be a platform for testing the motion control algorithms in a simulated pace which is faster than real-time in order for AI based approaches to be able to evolve within reasonable time frames. Either for tuning the settings of the motion control or for creating AI based algorithms for moving.

\subsection{Representation}
The presentation of the simulator will be shown with the position in three planes, which will give the user good tracking of the movement patterns of the AUV. To compliment this a three dimensional view of its orientation will give the user a clean and easy how the AUV turns.
