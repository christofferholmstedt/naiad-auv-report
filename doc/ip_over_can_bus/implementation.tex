% Do NOT change this "Section" title
% and do NOT add more "Section" level titles.
\section{Implementation}\label{sec:implementation}
The latest solution presented for IP over CAN is from Ditze et. al. with a 10
bits prioritisation field, 3 bits message type field, 8 bits destination
address and 8 bits sender address. This solution is shown in table
\ref{table:mapping_message_id_to_ip}. It's from this solution work started,
in the Naiad project.

\begin{table}[h]
\centering
    \caption{Mapping of message ID to IP over CAN bus.}
    \begin{tabular}{|c|c|c|c|} \hline
    \label{table:mapping_message_id_to_ip}
    Prioritisation & Msg & Sender Address & Dest. Address  \\ \hline
            10 bits & 3 bits & 8 bits & 8 bits \\ \hline
    \end{tabular}
\end{table}


In their solution Ditze et al. suggest a gateway that routes traffic to and from the
CAN bus network. The routing that has to be done would map full 32 bit IP
addresses to CAN bus IP addresses of 8 bits. This would be done by dropping the
first 24 bits of the IPv4 address and only using the last 8 bits over the CAN
Bus.

Ditze et al. don't go into
detail about assignment of IP addresses
but instead mention that it can be done statically or dynamically. As a goal
for the Naiad project was to to use Space plug-and-play Avionics technology
it introduce the requirement that low level address assignment had to be done
dynamically, to support plug-and-play features.
A first suggestion was to use a well-known message ID that all new
nodes would transmit a DHCP request with that message ID.
As the payload for the DHCP request, a host's identifier would be set.
The problem with this approach is that when multiple hosts boot up
simultaneously, a collision will occur and the CAN bus will go into an error
state.

The next approach was to use the destination address field and sender address field
with a total of 16 bits from the message ID as a unique host identifier field
when sending a DHCP request message (shown in table
\ref{table:mapping_message_id_to_ip_host_identifier}).
This approach seems to be viable and would
give 16 bit addresses which might be too few to prevent accidental CAN bus
collisions during boot up. The 16 bit identifier would have to be generated
from some kind of hardware ID such as the MAC Address for ethernet devices.

\begin{table}[h]
\centering
    \caption{Mapping of message ID to IP over CAN bus with Host identifier
    during DHCP process.}
    \begin{tabular}{|c|c|c|} \hline
    \label{table:mapping_message_id_to_ip_host_identifier}
    Prioritisation & Msg & Host identifier \\ \hline
            10 bits & 3 bits & 16 bits\\ \hline
    \end{tabular}
\end{table}

When worked had reached this point other priorities were made in the project so no further progress
was made.
