% Do NOT change this "Section" title
% and do NOT add more "Section" level titles.
\section{Conclusion}\label{sec:conclusion}
The approach taken during the Naiad project started from the solution presented
by Ditze et. al. This approach seemed to have some good features to it. Among
others the 10 bits prioritisation band seemed to be useful for systems with
hard real-time requirements. In the end it turned out that the same
prioritisation band might be too big which limits the number of bits that can be
used of host identification during the address assignment procedure. Future
work should look into the possibilty of using dynamic address assignment
specified by Cach et. al. while trying to keep a prioritisation band as big as
possible.

At the time of writing the latest solution in this field is over 10 years old
and put focus on IPv4. As of 2012 the IETF recommends that all IP capable
devices use IPv6. This recommendation should be followed and future work should
start from a IPv6 point of view.

Current focus around IPv6 is often targeting wireless communication which is
far from the CAN Bus specification but some parts are similiar. The targetted
wireless networks often run with small embedded 8-bit microcontrollers which is
often the case for hosts connected to CAN Busses as well. Future work should
take a closer look if any of the ideas behind protocols specified by 6LoWPAN
working group or the ROLL Working Group in IETF can be used for IP over CAN.


