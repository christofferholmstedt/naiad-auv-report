% Do NOT change this "Section" title
% and do NOT add more "Section" level titles.
\section{Lessons Learned}\label{sec:lessons_learned}

% You can use how many "subsections" and "subsubsections" you like.
\subsection{Development}
The requirements we got from our customers included one specific on the
programming language to use, Ada. In the beginning this felt like a bad choice
as the members of the project team had limited experience with Ada but we
couldn't do anything about it. Instead we took height for a longer learning
period and a more open mindset for future trouble such as which compilers would work
and for which architectures the Ada features we needed could be used on.

\subsubsection{Ada}
Ada is a strong-typed language. The first release of the language was published
in 1983 with updates in 1995, 2005 and the latest in 2012. With the limited
experience of Ada within the project team we didn't decide upon a specific version
to use early on. It later turned out that it would have been wise to limit us to
Ada 2005 functionality for better portability between different versions of
the compiler used.

\textbf{Compilation} and \textbf{cross-compilation} was done with the
"GNU Compiler Collection" (GCC) compiler including the GNAT frontend. When
someone is talking about "gnat" they are basically talking about GCC compiled
with Ada programming language support. As of January 2014 Ada support is not compiled
by default with GCC, though packages in Debian/Ubuntu exists for easy installation.

In the beginning of the project we were recommended to use the GNAT GPL compiler
supplied to academia for free by AdaCore. The latest release was from April
2013. The GNAT GPL compiler from AdaCore is basically the GCC compiler with
some extra features and bundled packages from AdaCore. The reason for the
recommendation was that the GNAT GPL version earlier included a compiler for
AVR architecture as well, this support was dropped with the 2013 release from
AdaCore. There were discussions made about aquiring a GNAT Pro version from AdaCore to
be able to compare functionality between the GPL and Pro version but the Pro 
version was never aquired.

As development platforms Windows 7/8 and Ubuntu 12.04 LTS/13.10 were used. Instead
of trying to aquire the GNAT Pro license we looked into the available packages
in the Ubuntu repository (those with Windows still used the AdaCore GNAT GPL version).
The GNAT packages available in Ubuntu is directly imported from the Debian
project and have a strict policy \cite{web:debian_ada_policy} how they should
be compiled so every Ada package available in the repository should work with all others.
The downside of this policy is that most packages are quite old but the upside
is that they all work together and are easy to install.

For cross-compilation against the AVR platform the AVR-Ada project \cite{web:avr-ada}
is available. It requires GCC 4.7.X and the supplied package in Ubuntu is gnat-4.6 so some
GCC compilation is required to be done to get it to work on Ubuntu.

We also used the BeagleBone Black in the project which is an ARM processor with
the ARMv7 architecture. As we installed Ubuntu directly on the BeagleBone Blacks
we could also install the Ubuntu package and compile Ada code natively directly
on the BeagleBone Blacks, we never tried to cross-compile for ARM.

In the end on Windows we use the AdaCore GNAT GPL bundle from 2012 to be able
to compile for the AT90CAN128 micro controller (AVR architecture). Compilation for
the BeagleBone Blacks was done with gnat-4.6 available in the repository, natively
on the board. As a lot of development and testing took place on Ubuntu 64-bits
machines both AdaCore GNAT GPL bundle from 2013 and the gnat-4.6 package from the
repository were used. One important aspect to remember here is that if the GPS
IDE is used remember to set the compiler to 2005 version otherwise some features
might compile on one machine and not on others. We noticed this when first trying
out our code on a development machine with the amd64 architecture and then compiling
the same code on the BeagleBone Blacks, it didn't work.

As final pointers it's worth mentioning that compiling for the AVR architecture
on Ubuntu is very limited, it works but support for specific MCUs is limited. During
Naiad project some development was made to support the AT90CAN128 but the work was
never finished. Also, compilation for the AVR architecture doesn't support any
object-oriented programming such as tagged types and interfaces nor Ada tasks, this
is true for both the AdaCore GNAT GPL bundle and the gnat-4.6 package from Ubuntu
repository.

\textbf{Online help} is available but very limited when it comes to Ada. Great
channels for Ada help are the newsgroup comp.lang.ada \cite{web:comp.lang.ada},
the \#Ada IRC channel on the Freenode network as well as related mailing lists
\cite{web:avr-ada-devel,web:debian-ada-devel}.

As recommended reading or at least reference litrature a few reports
exist on Ada development. These are the "Ada Style Guide" \cite{web:wikibooks-ada-style-guide},
the "Guide for the use of the Ada programming language in high integrity systems"
\cite{web:ada-high-integrity}, the "Guide for the use of the Ada Ravenscar Profile
in high integrity systems" \cite{web:ada-ravenscar-high-integrity}, the "High-Integrity
Object-Oriented Programming in Ada" reports \cite{web:ada-oop-high-integrity} and
the rationale behind Ada 2012 \cite{web:ada-rationale-2012}.

\subsubsection{Version Control System}
Available as a service from M\"{a}lardalens University is a restricted Subversion repository
including code and other work from previous projects in the course. The repository
size is pretty big so it's hard to work with. Early on in the project we decided
to go with Git and Github as version control system. Though only few of the project
members had used Git before so some training had to be done early on. This limitation
also made us choose a centralised version control system workflow. Basically
we set up a common repository on Github \cite{web:github_naiad-auv-software}
and gave all project members access to it.

The use of Git and Github was a good choice, though in the end we have had some
problems with the workflow e.g. that conflicts wasn't sorted out properly, especially
on Windows machines where the TortoiseGIT GUI can be hard to understand from time
to time. As suggestion for next year Git is the best choice for VCS though be prepared to
spend some extra time to teach those that don't know Git from before about a proper
distributed workflow such as git-flow \cite{web:git-flow} and put one project member
as release manager which is the only one that has commit access to the main/master branch.

The release manager will act as a guard to the main branch so only working, tested and
well written code is in the main branch. Another approach could be to look at
setting up your own installation of "Gerrit" \cite{web:gerrit} as a web-based peer-reviewing tool.
This would let everyone peer review everyone's code and you could limit merges
to the main branch to patches with at least 2 approved peer reviews.

\subsubsection{GPS with AUnit}
For unit testing AUnit was used. AUnit comes bundled with the AdaCore GPL version
and can be installed with the GNAT packages on Ubuntu. One main difference is that
the GPS IDE version that is bundled with AdaCore GPL has more AUnit functionality
than the version available in the Ubuntu repository.

\subsubsection{Coding Guidelines}
Early on in the project we sat down in the software team to decide upon coding
guidelines. The goal with coding guidelines is that code should be easy to read
and understand, primarly by other project members so they can help you as fast as
possible when you get stuck. Other project members should not have to spend hours
trying to parse your code.

No matter what guidelines you chose someone has
to enforce these otherwise project members will start to say that they will fix
that later on, but that will
never happen. Before deciding upon your own coding guidelines it's wise to look
at the public "Ada Quality and Style Guide" available on Wikibooks \cite{web:wikibooks-ada-style-guide}.

\subsubsection{Technical Reports in LaTeX}
Just as keeping track of different versions of your Ada code, LaTeX text can be tracked
with Git as well. One thing to remember is to use linebreaks when rows starts to
get too long. The reason for this is that if a line consists of a complete paragraph
and you change just a word in that paragraph Git will show that the entire paragraph
has changed because it's written in a single line. If you instead break the paragraph
with several linebreaks and then change a single word, only that line the tex file
will be marked as changed.

\subsection{Operations}
"Operations" consists of all the software and hardware we had to maintain during the
project. This includes some network equipment and a server machine the project room
as well as the webserver installed on a virtual private server.

\subsubsection{Continious Integration with AUnit}
As mentioned earlier, AUnit was used for unit testing and the idea was to set up
our server machine with some kind of continous integration testing server. After
each commit to the Git repository the test server would run all tests and give
a positive or negative response back if all tests passed or not. If any test failed
the developer would be informed with details about which test didn't pass.

Jenkins CI \cite{web:jenkins} was set up though we didn't get any public IP address so we had to use
a polling scheme for the test server so it checked for new code every 5 minutes.
The lack of public IP also restricted us to only get access to the test results
when we were at the university. The Jenkins service was up and running throughout the
project but it wasn't used at all. In the end, the lesson learned was that
if project members can't see the benefits
of a service, it's not worth maintaining it.

\subsubsection{Website}
For the website it was decided that we wanted to use some kind of content management
system and not develop it from scratch, time is better spent on other tasks.
The experience within the project team suggested that we would go with Wordpress.

Wordpress was set up on a Ubuntu virtual machine at the hosting company Tilaa
located in the Netherlands \cite{web:tilaa}. The price for the VPS was about 17 euro
a month, including VAT. The rationale for this was that we could have used a service such
as wordpress.com but end up with a lot of extra costs for changing the design as we
want it. Wordpress.com is a good service when you don't want to change too much.
Our own VPS gave us much more freedom in the design of the website.

For tracking visitors and the number of hits we used "Piwik" \cite{web:piwik}.

\subsubsection{Backup}
For backup of the website we used daily cronjobs on the server machine in the project
room to pull down the current information available on the VPS.
