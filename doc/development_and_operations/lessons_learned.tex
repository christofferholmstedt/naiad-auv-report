% Do NOT change this "Section" title
% and do NOT add more "Section" level titles.
\section{Lessons Learned}\label{sec:lessons_learned}

% You can use how many "subsections" and "subsubsections" you like.
\subsection{Development}
The requirements we got from our customers including one specific on the
programming language to use, Ada. In the beginning this felt like a bad choice
as the members of the project team had limited experience with Ada but as we
couldn't do anything about it. Instead we took height for a longer learning
period and a more open mind set for future trouble such as which compilers works
and for which architectures all Ada features can be used on.

\subsubsection{Ada}
Ada is a strong-typed language. The first release of the language was published
in 1983 with updates in 1995, 2005 and the latest in 2012. With the limited
experience of Ada within the project team we didn't decide upon a specific version
to use early on. It later turned out that it would have been wise to limit us to
Ada 2005 functionality for better portability between different version of
the compiler used.

\textbf{Compilation} and \textbf{cross-compilation} was done with the
"GNU Compiler Collection" (GCC) compiler including the GNAT frontend. When
someone is talking about "gnat" they are basically talking about GCC compiled
with Ada programming language support. As of January 2014 ada is not compiled
by default with GCC though packages in Debian/Ubuntu exists for easy installation.

In the beginning of the project we were recommended to use the GNAT GPL compiler
supplied to academia for free by "Adacore". The latest release was from April
2013. The GNAT GPL compiler from Adacore is basically the GCC compiler with
some extra features and bundled packages from Adacore. The reason for the
recommendation were that the GNAT GPL version earlier included a compiler for
AVR architecture as well, this support was dropped with the 2013 release from
Adacore. Some talks were made about aquiring a GNAT Pro License from Adacore to
be able to compare functionality between the GPL and PRO license but it was left
with just talks.

As development platforms Windows 7/8 and Ubuntu 12.04 LTS/13.10 were used. Instead
of trying to aquire the GNAT Pro license we looked into the available packages
in the Ubuntu repository (those with Windows still used the Adacore GNAT GPL version).
The GNAT packages available in Ubuntu is directly imported from the Debian
project and have a strict policy \cite{web:debian_ada_policy} how they should
be compiled so every Ada package available in the repository should work with all others.
The downside of that policy is that most packages are quite old but the upside
is that they all work together and are easy to install.

For cross-compilation against the AVR platform the AVR-Ada project \cite{web:avr-ada}
is available. It requires GCC 4.7.X and the supplied package in Ubuntu is gnat-4.6 so some
gcc compilation is required to be done to get it to work on Ubuntu.

We also used the BeagleBone Black in the project which is an ARM processor with
the ARMv7 architecture. As we installed Ubuntu directly on the BeagleBone Blacks
we could also install the Ubuntu package and compile Ada code natively directly
on the BeagleBone Blacks, we never tried to cross-compile for ARM.

In the end on Windows we use the Adacore GNAT GPL bundle from 2012 to be able
to compile for the at90can micro controller (AVR architecture). Compilation for
the BeagleBone Blacks was done with gnat-4.6 available in the repository, natively
on the board. As a lot of development and testing took place on Ubuntu 64-bits
machines both Adacore GNAT GPL bundle from 2013 and the gnat-4.6 package from the
repository were used. One important aspect to remember here is that if the GPS
IDE used remember to set the compiler to 2005 version otherwise some features
might compile on one machine and not on others. We noticed this when first trying
out our code on a development machine with the amd64 architecture and then compiling
the same code on the BeagleBone Blacks, it didn't work.

As some final pointers it's worth mentioning that compiling for the AVR architecture
on Ubuntu is very limited, it works but support for specific MCUs is limited. During
Naiad project some development was made to support the at90can but the work was
never finished. Also, compilation for the AVR architecture doesn't support any
object-oriented programming such as tagged types and interfaces nor Ada tasks, this
is true for both the Adacore GNAT GPL bundle and the gnat-4.6 package from Ubuntu
repository.

\textbf{Online help} is available but very limited when it comes to Ada. Great
channels for Ada help are the newsgroup comp.lang.ada \cite{web:comp.lang.ada},
the \#Ada IRC channel on the Freenode network as well as related mailing lists
\cite{web:avr-ada-devel,web:debian-ada-devel}.

\subsubsection{Version Control System}
Available as a service from M\"{a}lardalens University is a restricted Subversion repository
including code and other work from previous projects in the course. The repository
size is pretty big so it's hard to work with. Early on in the project we decided
to go with Git and Github as version control system. Though only few of the project
members had used Git before so some training had to be done early on. This limitation
also made us choose a centralised version control system workflow. Basically
we set up a common repository on Github \cite{web:github_naiad-auv-software}
and gave all project members access to it.

The use of Git and Github was a good choice in the end though we have had some
problems with the workflow that conflicts wasn't sorted out properly, especially
on Windows machines where the TortoiseGIT GUI can be hard to understand from time
to time. As suggestion for next year Git is the best choice for VCS though be prepared to
spend some extra time to teach those that don't know Git from before about a proper
distributed workflow such as git-flow \cite{web:git-flow} and put one project member
as release manager which is the only one that has commit access to the main/master branch.

\subsubsection{Continious Integration with AUnit}
jenkins build server

\subsubsection{Coding Guidelines}
Decide upon a first set of guidelines and let the guidelines evolve

\subsubsection{LateX}
Use git
use line breaks.
Text

\subsection{Operations}
No public ip

\subsubsection{VPS - Tilaa}
Text
\subsubsection{Wordpress and MySQL}
Text
\subsubsection{Backup}
website
wiki

