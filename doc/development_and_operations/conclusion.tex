% Do NOT change this "Section" title
% and do NOT add more "Section" level titles.
\section{Conclusion}\label{sec:conclusion}
A lot can be done with different tools to improve the efficiency of the development
team and the quality of the code that is produced, though it's important to remember
that too much of it will leave it all unused. Tools that are in the daily workflow
are the ones that will be worth installing and maintaining.

\subsection{Development}
\subsubsection{Ada}
Try to stick with one development environment and make sure to set up cross-compilation
of all your code straight away. This will prevent surprises were you end up rewriting
a lot of code that works on the development machines but not on the ARM or AVR
architectures.

To make sure all project members run with the same development environment you
could look in to the possibility of setting a Vagrant \cite{web:vagrant}
virtual machine system and letting everyone run with that.

\subsubsection{Version Control System}
Git with Github services was a good choice for version control system though
the workflow chosen was not. The time spent learning a distributed workflow such
as git-flow is time well spent early on in the project.

One last thing
to remember is that a repository can grow very quickly if you allow binary files
to be added. Some test files such as test images for the vision system are needed
to be in the repository but try to limit it to as few as possible. This will
keep the repository size to a minimum and make it easier to clone on new
development machines.

\subsection{Operations}
\subsubsection{Website}
If possible try to use wordpress.com service or similiar. The burden of
maintaining the website is not worth it and you will save some well needed cash 
as well. A public service such as wordpress.com
will also make it easier to transfer the responsibility to others when the project
course is over. The downside is that you won't have that much of flexibility when
it comes to design.
