% Do NOT change this "Section" title
% and do NOT add more "Section" level titles.
\section{Result and future work}\label{sec:result}

\subsection{Ravenscar profile restrictions}
The system is implemented to compile with ravenscar profile restrictions \cite{article:mcsraven} and successfully does so.

\subsection{Receiving missions}
The system's ability to receive missions through TCP and start executing them is finished and tested.

\subsection{CAN communication}
The system currently lacks the functionality to interpret CAN messages sent from other systems in the AUV. The system is implemented to the point where it can receive and send CAN messages, however there is no functionality for handling the incoming messages.

% You can use how many "subsections" and "subsubsections" you like.
\subsection{Virtual machine}
The virtual machine is implemented to the point where it interprets and executes any stack machine code compiled from NaiAda source code. However, it lacks functionality for communicating with the other systems in the AUV. Stack machine code instructions that are to be used for communication need to be implemented.

\subsection{Integrated development environment}
The IDE is still under development. The IDE also needs primitives in order to be usable, which need to be written in NaiAda.

\subsection{Aggregator}
The python script which uses XML files created by the IDE to aggregate NaiAda source code files into one file is finished and tested.

\subsection{Compiler}
The compiler is finished and tested. It compiles correctly written NaiAda code produced by the aggregator into stack machine code and reports naming errors, type errors and syntax errors.

\subsection{Byte-code converter}
The byte-code converter is finished and tested. It's a small program implemented in Ada, which takes a file containing stack machine code in text, and produces a binary object file which can be run on the VM as a mission.