% Each section is a new appendix
\section{Appendix}
\subsection{Getting started with GIMME-2 board}\label{sec:first_appendix}
This section describes the step by step procedure to get started with the GIMME-2 board. The operating system used on the host machine is Ubuntu 12.04. The prerequisites for initializing the GIMME-2 board are:\begin{enumerate}
\item Micro-USB Male (Type B) to USB Male (Type A) connector or an Ethernet cable.
\item Power supply. The board should be supplied with voltages in the range +12V to +24V.
\item If the user is planning to use the USB cable, then a serial terminal, for example \textit{gtkterm}, should be installed on the host PC to interact with the Linux on the GIMME-2 board.
\end{enumerate}

Assuming that Zynq Linux is already installed on the GIMME-2 board, the user can interact with it in either of the two ways described below. Ensure that the jumper configuration is 0100 before powering on the board.
\subsubsection{Using serial interface}
Connect the micro-USB to USB connector to the port labelled \textit{USB0} on the board and connect the other end to the host PC. Set the voltage of the power supply to +12V and connect it to the GIMME-2 board. Now open the serial terminal and the booting steps can be seen there. Once the booting is over, the user gets a \texttt{zynq>} prompt on the screen.

\subsubsection{Using Ethernet}
Connect one end of the Ethernet cable to the middle port among the three on the GIMME-2 board and the other end to the host machine. Now open the bash terminal and execute
\texttt{ssh root@192.168.1.10}. Type in \textit{root} when password is prompted. The user gets a \texttt{zynq>} prompt on the screen.

\subsection{Creation of Zynq Boot Image}
The procedure for creating a boot image is quite complex and time consuming. For the sake of the new users, all the files required to create a boot image are available on the Subversion repository for project Naiad \cite{web:svnGimme}. Additional to the prerequisites mentioned in section 5.1, the user needs to do the following:\begin{enumerate}
\item Install the full package of Xilinx ISE Design Suite\cite{web:downloadISE} including the Software Development Kit (SDK) on the host PC.
\item Install the driver for Xilinx Platform USB Cable II \cite{web:driverCable}. If Step 1 is done in Windows OS, then no need to install the driver as it comes with the package.
\end{enumerate}

Using SDK, new boot image can be created by giving offsets to the binary files as mentioned in the user guide \cite{web:svnGimme}. If amendments have to be made to the Linux kernel like installing a new Linux driver, then the RAM disk image should be modified and a new boot image should be created with the new RAM image. All the steps for modifying RAM image are also detailed in the user guide \cite{svnGimme}.
